\documentclass[letterpaper]{article} 
\usepackage[left = 0.5in, right = 0.5in, top = 0.9in, bottom = 0.9in]{geometry}
\usepackage{enumitem}
\usepackage{multicol}
\usepackage[spanish]{babel}
\usepackage[utf8]{inputenc}

\usepackage{amsmath,amssymb,amsthm}
\usepackage{tikz-cd}
\usepackage{mathrsfs}
\usepackage[bbgreekl]{mathbbol}
\usepackage{dsfont}
\usepackage{graphicx}
\graphicspath{{img/}}

\newcommand{\op}{\operatorname}
\newcommand{\Op}{^{\op{op}}}
\newcommand{\scc}{\mathscr C}
\newcommand{\scd}{\mathscr D}
\newcommand{\sce}{\mathscr E}
\newcommand{\sci}{\mathscr I}
\newcommand{\scj}{\mathscr J}
\newcommand{\scx}{\mathscr X}
\newcommand{\var}{\mathrm{Var}}
\newcommand{\Id}{\operatorname{Id}}
\newcommand{\N}{\mathbb N}
\newcommand{\Z}{\mathbb Z}
\newcommand{\Q}{\mathbb{Q}}
\newcommand{\I}{\mathbb{I}}
\newcommand{\R}{\mathbb{R}}
\newcommand{\C}{\mathbb{C}}
\newcommand{\F}{\mathcal{F}}
\newcommand{\G}{\mathcal{G}}
\newcommand{\B}{\mathcal{B}}
\newcommand{\abs}[1]{\left\lvert #1 \right\rvert}
\newcommand{\inv}{^{-1}}
\renewcommand{\to}{\rightarrow}
\newcommand{\ent}{\Longrightarrow}
\newcommand{\E}{\mathbb{E}}
\renewcommand{\P}{\mathbb{P}}
\newcommand{\1}{\mathds{1}}
\renewcommand{\qedsymbol}{$\blacksquare$}

\theoremstyle{definition}
\newtheorem{dfn}{Definición}
\theoremstyle{definition}
\newtheorem{teo}{Teorema}
\theoremstyle{definition}
\newtheorem{cor}{Corolario}
\theoremstyle{definition}
\newtheorem{prop}{Proposición}
\theoremstyle{definition}
\newtheorem{obs}{Observación}


\title{\textbf{Modelos Estocásticos en Finanzas\\
Tarea-examen 3}}
\author{Iván Irving Rosas Domínguez}
\date{\today}

\DeclareSymbolFontAlphabet{\mathbbm}{bbold}
\DeclareSymbolFontAlphabet{\mathbb}{AMSb}
\DeclareMathSymbol\bbDelta  \mathord{bbold}{"01}

\begin{document}
\maketitle

%\begin{abstract}
%\end{abstract}

\begin{itemize}
    \item[\textbf{1.}] Considere la función $v_L$ dada por 
    \begin{equation}\label{eq1}
        v_L(x)=
        \begin{cases}
            K-x & \text{ si } 0\leq x \leq L,\\
            (K-L)\left(\frac{x}{L}\right)^{-2r/\sigma^2} & \text{ si } x\geq L.
        \end{cases}    
    \end{equation}
    La primera línea de la ecuación anterior implica que $v_L'(L-)=-1$. Use la segunda 
    línea de \eqref{eq1} para calcular $v_L'(L+)$. Muestre que el \textit{smooth pasting} dado por 
    \[
    v_{L_*}'(L_*-)=v_{L_*}'(L_*+),
    \]
    se satisface solo por $L_*$ dado por 
    \begin{equation}\label{eq6}
        L_*=\frac{2r}{2r+\sigma^2}K.
    \end{equation}
    \begin{proof} 
      Calculamos la derivada de $v_L$ dada por \eqref{eq1} en el intervalo $[L,\infty)$. Nótese que $L>0$ por lo que las expresiones 
      de la segunda línea están bien definidas, así que para $x>L$, 
      \[
        v_L'(x)=(K-L)\left(-\frac{2r}{\sigma^2}\right)\left(\frac{x}{L}\right)^{-2r/\sigma^2-1}\left(\frac{1}{L}\right)=-\frac{2r(K-L)}{L\sigma^2}\left(\frac{x}{L}\right)^{-2r/\sigma^2-1}.
      \]
      Nótese que lo anterior es una función continua de $x$, ya que $x>L>0$. Luego, para calcular la derivada por derecha, hacemos
      tender $x$ a $L$ por derecha, lo que nos dice que 
      \[
      v_L'(L+)=-\frac{2r(K-L)}{L\sigma^2}\left(\frac{L}{L}\right)^{-2r/\sigma^2-1}=-\frac{2r(K-L)}{L\sigma^2}.
      \]
      Se sigue que, para que las derivadas por derecha e izquierda de $v_L$ coincidan en $L$, se debe tener que 
      \[
      v_L'(L+)=v_L'(L-) \ \iff  \ -1= -\frac{2r(K-L)}{L\sigma^2},
      \]
      por lo que resolviendo para $L$, se tiene que 
      \begin{align*}
        -1= -\frac{2r(K-L)}{L\sigma^2} &\iff 1= \frac{2r(K-L)}{L\sigma^2}\\
        &\iff L\sigma^2= 2rK-2rL\\
        &\iff L(\sigma^2+2r)= 2rK\\
        &\iff L= \frac{2r}{2r+\sigma^2}K\\
      \end{align*}
      por lo que el \textit{smooth pasting}, se logra siempre que 
      \[
      L_*=  \frac{2r}{2r+\sigma^2}K.
      \]
     \end{proof}
    \item[\textbf{2.}] Considere dos puts Americanos perpetuos con base en un movimiento 
    browniano geométrico dado por 
    \[
    dS(t)=rS(t)dt+\sigma S(t)d\widetilde{W}(t)    
    \]
    Supongamos que los puts tienen diferentes precios de strike, $K_1$ y $K_2$, donde $0<K_1<K_2$. Sean 
    $v_1(x)$ y $v_2(x)$ sus respectivos precios (como se calculó en clase). Muestre que $v_2(x)$ satisface 
    las primeras dos condiciones lineales 
    \begin{eqnarray}
        v_2(x)\geq (K_1-x)^{+} & \text{ para todo } x\geq 0,\label{eq2}\\ 
        rv_2(x)-rxv_2'(x)-\frac{1}{2}\sigma^2x^2v_2''(x)\geq 0 & \text{ para todo } x\geq0\label{eq3}, 
    \end{eqnarray}   
    para el precio del put Americano perpetuo con precio de strike $K_1$ pero que $v_2(x)$ no satisface la 
    tercera condición dada por: 
    \[
    \text{Para toda } x\geq0 \text{ la igualdad se da en \eqref{eq2} o en \eqref{eq3} o en ambas.}    
    \]  
    \begin{proof} 
    Recordamos que, de acuerdo a lo visto en clase, para el precio $K_1$ y $K_2$, las funciones $v_1$ y $v_2$ están 
    dadas por
    \[v_i(x)= 
    \begin{cases}
        (K_i-x) & \text{ si } 0\leq x \leq L\\
        (K_i-L)\left(\frac{x}{L}\right)^{-\frac{2r}{\sigma^2}} & \text{ si } L\leq x,
    \end{cases}
    \] 
    por lo que vemos que $v_2$ cumple las primeras dos condiciones $\eqref{eq2}$ y $\eqref{eq3}$:
    \begin{itemize}
        \item Nótese que para $x\geq0$, si $0\leq x\leq L$, entonces directamente de la definición 
        de $v_2(x)$ se tiene que $v_2(x)=K_2-x$, luego, como $0<K_1<K_2$, se tiene que 
        \[
        v_2(x)\geq K_1-x=(K_1-x)^{+},    
        \]
        donde la última igualdad se debe a que el valor $L$ el valor de la barrera inferior al precio de strike $K$ se escoge justamente 
        tal que $0<L<K_1<K_2$. 
        Por lo tanto, en $[0,L]$ se cumple la condición $\eqref{eq2}$. Supongamos ahora que 
        $L\leq x$. Nótese así que
        \[
            v_2(x)=(K_2-L)\left(\frac{x}{L}\right)^{-\frac{2r}{\sigma}^2}\geq (K_2-L)\left(1\right)^{-\frac{2r}{\sigma^2}}=K_2-L.
        \]
        Luego, si $K_1-x\geq0$, entonces $K_2-L\geq K_1-x\geq0$, y con ello, $v_2(x)\geq (K_1-L)^{+}$.
        Por otro lado, si $K_1-x\leq0$, entonces $K_2-L\geq0$ pues $0\leq L< K_1<K_2$ y con ello, $v_2(x)=K_2-L\geq0=(K_1-x)^{+}$.
        En cualquier caso, se cumple la primera condición \eqref{eq2}.
        \item Supongamos que $0\leq x\leq L$. Es claro en este caso que las primeras y segundas derivadas de $v_2$ están 
        dadas por 
        \[
            v'_2(x)=-1 \qquad \text{ y } \qquad v'_2(x)=0,
        \]
        por lo que la ecuación \eqref{eq3} está dada para $v_2$ por 
        \[
        r(K_2-x)-rx(-1)-0=rK_2-rx+rx=rK_2\geq0, 
        \]
        por lo que se cumple la condición \eqref{eq3}. Supongamos ahora que $L\leq x$, entonces 
        las derivadas para $v_2(x)$ en este intervalo están dadas por 
        \[
            v_2'(x)=-\frac{2r}{\sigma^2}(K_2-L)L^{2r/\sigma^2}x^{-2r/\sigma^2-1},
        \]
        y 
        \[
            v_2''(x)=-\frac{2r}{\sigma^2}\left(-\frac{2r}{\sigma^2}-1\right)(K_2-L)L^{2r/\sigma^2}x^{-2r/\sigma^2-2},
        \]  
        por lo que entonces la condición de la \eqref{eq3} se vuelve:
        \[
          r(K_2-L)x^{-\frac{2r}{\sigma^2}}L^{\frac{2r}{\sigma^2}}-rx \left(-\frac{2r}{\sigma^2}(K_2-L)L^{2r/\sigma^2}x^{-2r/\sigma^2-1}\right)-\frac{1}{2}\sigma^2x^2 \left(-\frac{2r}{\sigma^2}\right) \left(-\frac{2r}{\sigma^2}-1\right)(K_2-L)L^{2r/\sigma^2}x^{-2r/\sigma^2-2},
        \]
        la cual al simplificar se vuelve 
        \[
            r(K_2-L)x^{-\frac{2r}{\sigma^2}}L^{\frac{2r}{\sigma^2}}+r\frac{2r}{\sigma^2}(K_2-L)L^{2r/\sigma^2}x^{-2r/\sigma^2}+r\left(-\frac{2r}{\sigma^2}-1\right)(K_2-L)L^{2r/\sigma^2}x^{-2r/\sigma^2},
        \]
        y al factorizar el término común $r(K_2-L)x^{-\frac{2r}{\sigma^2}}L^{\frac{2r}{\sigma^2}}$, obtenemos que 
        \[
            x^{-\frac{2r}{\sigma^2}}L^{\frac{2r}{\sigma^2}} \left(1+\frac{2r}{\sigma^2}-1-\frac{2r}{\sigma^2}\right)=0\geq0,   
        \]
        por lo que en efecto se cumple la segunda condición \eqref{eq3}. 
        \item Resta que veamos que NO se cumplen simultáneamente \eqref{eq2} y \eqref{eq3} (o ambas). Nótese que para ello, basta verificar 
        la existencia de un $x\in [0,\infty)$ tal que no se cumplan ni la igualdad en \eqref{eq2} ni la igualdad en \eqref{eq3}. Proponemos 
        $x=\frac{L}{2}$. Es claro que entonces $0<x<L<K_1<K_2$. Notamos que no se da la igualdad en \eqref{eq3}: como $0<x<L$, entonces $v_2(x)=K_2-x$ y así,
        según las derivadas calculadas antes,
        \[
        rv_2(x)-rxv_2'(x)-\frac{1}{2}\sigma^2x^2v_2''(x)=r(K_2-x)-rx(-1)-0=rK_2-rx+rx=rK_2,
        \]
        y como $r,K_2>0$ por hipótesis, entonces se da la desigualdad estricta en \eqref{eq3}. Por otro lado, 
        nótese que tampoco se da la igualdad en \eqref{eq2}, esto ya que 
        \[
        v_2(x)=K_2-x.    
        \]
        Mientras que 
        \[   
            (K_1-x)^{+}=K_1-x,
        \]
        porque $0<L/2=x<L<K_1<k_2$. Se sigue que, si se diera la igualdad en \eqref{eq2}, entonces
        \[
            K_2-x=K_1-x \iff K_2=K_1,
        \]
        lo cual es una contradicción. Por lo tanto, no se da la igualdad en \eqref{eq2} y con ello, 
        el precio $v_2$ de la opción con precio de strike $K_2>K_1$ no satisface la tercera de las condiciones 
        lineales complementarias para el precio de strike $K_1$.
    \end{itemize}
     \end{proof}
    \item[\textbf{3.}] Suponga que $v(x)$ es una función continua y acotada con derivada continua 
    y que satisface 
    \begin{eqnarray}
        v(x)\geq (K-x)^{+} & \text{ para todo } x\geq 0,\label{eq4}\\ 
        rv(x)-rxv'(x)-\frac{1}{2}\sigma^2 x^2v''(x)\geq 0 & \text{ para todo } x\geq0,\label{eq5}\\
        \text{Para toda } x\geq0 \text{ la igualdad se da en \eqref{eq4} o en \eqref{eq5}.}
    \end{eqnarray}
    Este ejercicio muestra que $v(x)$ tiene que ser $v_{L_*}(x)$ dada por $\eqref{eq1}$ con $L_*$ dado por \eqref{eq6}.
    Suponga que $K$ es estrictamente positiva. 
    \begin{enumerate}
        \item Primero, considérese un intervalo de valores en $x$ en el que $v(x)$ satisface 
        
        \begin{equation}\label{eq7}
                rv(x)-rxv'(x)-\frac{1}{2}\sigma^2x^2v''(x)=0.    
        \end{equation}
        
        Esta es una ecuación diferencial de segundo orden lineal, y tiene dos soluciones de la forma 
        $x^{p}$. Sustituya $x^p$ en \eqref{eq7} y muestre que los únicos valores de $p$ que producen que $x^{p}$ satisfaga
        \eqref{eq7} son $p=-\frac{2r}{\sigma^2}$ y $p=1$.
        \begin{proof} 
          Supongamos que se satisface \eqref{eq7}. Sustituyendo $v(x)=x^{p}$, se tiene que 
          \[
            rx^{p}-rxpx^{p-1}-\frac{1}{2}\sigma^2x^2p(p-1)x^{p-2}=0 \iff rx^{p}-rpx^{p}-\frac{1}{2}\sigma^2p(p-1)x^{p}=0,
          \]
          y ahora, suponiendo que $x=0$ es un valor dentro del intervalo, se tiene la igualdad anterior. Supongamos ahora que $x\neq0$. Entonces cancelando 
          términos y resolviendo para $p$, se tiene que 
          
          \begin{align*}
            r-rp-\frac{1}{2}\sigma^2p(p-1)=0 &\iff p^2+p(\frac{2r}{\sigma^2}-1)=\frac{2r}{\sigma^2}\\
            &\iff p^2+2p \left(\frac{r}{\sigma^2}-\frac{1}{2}\right)+\left(\frac{r}{\sigma^2}-\frac{1}{2}\right)^2=\frac{2r}{\sigma^2}+\left(\frac{r}{\sigma^2}-\frac{1}{2}\right)^2\\
            &\iff \left(p+\frac{r}{\sigma^2}-\frac{1}{2}\right)^2=\frac{2r}{\sigma^2}+\left(\frac{r}{\sigma^2}-\frac{1}{2}\right)^2\\
            &\iff \left(p+\frac{r}{\sigma^2}-\frac{1}{2}\right)^2=\frac{r^2}{\sigma^4}+2\frac{r}{\sigma^2}\left(\frac{1}{2}\right)+\frac{1}{4}\\
            &\iff \abs{p+\frac{r}{\sigma^2}-\frac{1}{2}}=\abs{\frac{r}{\sigma^2}+\frac{1}{2}}\\
            &\iff \begin{cases}
                p+\frac{r}{\sigma^2}-\frac{1}{2}=\frac{r}{\sigma^2}+\frac{1}{2} \qquad \iff \qquad p=1\\
                p+\frac{r}{\sigma^2}-\frac{1}{2}=-\frac{r}{\sigma^2}-\frac{1}{2} \qquad \iff \qquad p=-\frac{2r}{\sigma^2}\\
                \end{cases}
          \end{align*}
          \end{proof}
        \item Las funciones $x^{-\frac{2r}{\sigma^2}}$ y $x$ son soluciones linealmente independientes 
        de $\eqref{eq7}$, entonces cualquier solución de $\eqref{eq7}$ debe ser de la forma 
        \[
        f(x)=Ax^{-2r/\sigma^2}+Bx,    
        \]
        para algunas constantes $A$ y $B$. Use este hecho y el hecho de que $v(x)$ y $v'(x)$ son continuas
        para mostrar que no puede existir un intervalo $[x_1,x_2]$ con $0<x_1<x_2<\infty$, tal que 
        $v(x)$ satisface $\eqref{eq7}$ en $[x_1,x_2]$ y satisface 
        
        \begin{equation}\label{eq8}
            v(x)\geq (K-x)^{+}    
        \end{equation}
        con igualdad para $x$ en $[x_1,x_2]$ e idénticamente a la izquierda de $x_1$ y para $x$ en inmediatamente 
        a la derecha de $x_2$ a menos que $v(x)$ sea idénticamente 0 en $[x_1,x_2]$.
        \begin{proof} 
         Supongamos que $x_1,x_2$ son números tales que $[x_1,x_2]\subseteq (0,\infty)$ y tales que 
         $v(x)$ cumple \eqref{eq7} y \eqref{eq8} con igualdad. Esto nos dice que $v(x)=Ax^{-2r/\sigma^2}+Bx$ para algunas constantes
         $A$ y $B$. Ahora bien, supongamos que el intervalo $[x_1,x_2]$ es tal que su intersección con $[K,\infty)$ es no nula. Se sigue 
         entonces que existen puntos en $[x_1,x_2]$ en donde $x\geq K$, y en consecuencia, $(K-x)^{+}=0$ por lo que la condición 
         \eqref{eq8} con igualdad se vuelve $Ax^{-2r/\sigma^2}+Bx=0$, lo cual nos dice que 
         $Ax^{-2r/\sigma^2}=-Bx \iff Ax^{-2r/\sigma^2-1}=-B$, de tal forma que una función de $x$, que 
         puede tomar infinitos valores (salvo el caso en que $x_2=K$) es igual a una constante. Por lo tanto, 
         para que la igualdad se de, se debe cumplir que $A=B=0$ y con ello $v$ es idénticamente 0 en $[x_1,x_2]$.
         \newline
         
         Finalmente, suponiendo que no ocurre lo anterior y que $[x_1,x_2] \subseteq [0,K]$, entonces $(K-x)^{+}=K-x$, 
         de tal forma que la condición \eqref{eq8} con igualdad se vuelve 
         \[
            Ax^{-2r/\sigma^2}+Bx=K-x \iff Ax^{-2r/\sigma^2}=K-(B+1)x,   
         \]
         y analizando nuevamente la ecuación anterior, nos damos cuenta que, al ser $\frac{2r}{\sigma^2}$ un número positivo, 
         se tiene que una curva del tipo hipérbola está igualada a una recta con pendiente negativa y ordenada al origen $K$,
         por lo que a lo más, se podrían intersecar en dos puntos. Pero la igualdad anterior ocurre para cualquier punto 
         en $[x_1,x_2]\subseteq[0,K]$, entonces la igualdad ocurre solamente si $A=B=0$ y nuevamente $v$ es idénticamente 0. La continuidad de $v$ y su derivada $v'$' 
         nos asegura que este comportamiento se preserva en los extremos del intervalo $[x_1,x_2]$.
        \end{proof}
        \item Use el hecho de que $v(0)$ tiene que ser igual a $K$ para mostrar que no puede existir un número $x_2>0$ 
        tal que $v(x)$ satisface 
        \begin{equation}\label{eq9}
            rv(x)-rxv'(x)-\frac{1}{2}\sigma^2x^2v''(x)\geq0
        \end{equation}
        con igualdad en $[0,x_2]$.
        \begin{proof} 
          Supongamos que tal número $x_2>0$ existe y nótese que, si $v(x)$ satisface \eqref{eq9} con igualdad en $[0,x_2]$, 
          entonces por el inciso anterior, $v$ debe ser de la forma $v(x)=Ax^{-2r/\sigma^2}+Bx$, la cual 
          es una función que se anula en 0. Luego, $v(0)=0$, y este comportamiento 
          se preserva en los extremos del intervalo $[0,x_2]$ de acuerdo al ítem anterior. Pero sabemos que $v(0)=K>0$, 
          por lo que no es posible que exista $x_2>0$ con dicha propiedad.  
         \end{proof}
        \item Explique por qué $v(x)$ no puede satisfacer $\eqref{eq8}$ con igualdad para todo $x\geq0$.\\
        
         \textbf{Solución:} Esto se explica porque, de ser el caso que \eqref{eq8} se cumpla en todo $[0,\infty)$, entonces 
         $v(x)=(K-x)^{+}$, la cual no es una función derivable en el punto $x=K$. O bien, aún ignorando que 
         la función no sea derivable en ese punto, la derivada de la misma no sería continua. Por lo tanto, 
         no puede darse la igualdad en todo $[0,\infty)$. 

        \item Explique por qué $v(x)$ no puede satisfacer $\eqref{eq9}$ con igualdad para todo $x\geq0$.\\
        
        \textbf{Solución:} Esto se explica justamente en el ítem 3, ya que de satisfacerse \eqref{eq9} con igualdad 
        para todo $x\geq0$, en particular se satisface para algún intervalo $[0,x_2]$, con $x_2>0$, pero esto no es posible.\\

        \item De los dos ítems anteriores y la condición \textit{Para toda $x\geq0$, la igualdad se da en \eqref{eq8} o en \eqref{eq9}}. Vemos que 
        $v(x)$ a veces satisface $\eqref{eq8}$ con igualdad y otras no la satisface con igualdad, en 
        cuyo caso debe satisfacer $\eqref{eq9}$ con igualdad. Del ítem 2 y 3 vemos que la 
        región en la cual $v(x)$ no satisface $\eqref{eq8}$ con igualdad y satisface $\eqref{eq9}$ con igualdad 
        no puede ser un intervalo $[x_1,x_2]$ con $0<x_1<x_2<\infty$, y tampoco puede ser una unión 
        disjunta de intervalos de esta forma. Entonces tiene que ser un semi-eje $[x_1,\infty)$, con $x_1>0$. En 
        la región $[0,x_1]$, $v(x)$ satisface $\eqref{eq8}$ con igualdad. Muestre que $x_1$ tiene que ser igual 
        a $L_*$ y que $v(x)$ tiene que estar dada por $v_{L_*}(x)$ dada por $\eqref{eq1}$.
        \begin{proof} 
          Notamos que se tiene que satisfacer $x_1=L_*$. En efecto. Dado que $v$ tiene que ser una función 
          derivable con derivada continua y acotada, se tiene que tener que $v(x)=(K-x)^{+}$ debe coincidir 
          por derecha con $v(x)$ dada por $Ax^{-\frac{2r}{\sigma^2}}+Bx$, y a su vez, la derivada 
          de $v(x)$ tiene que ser continua, por lo que las expresiones de \eqref{eq8} y \eqref{eq9} y sus derivadas 
          deben ser iguales. Nótese que imponiendo las condiciones de consistencia, tenemos que 
          \[
          v(x)=K-x=Ax^{-2r/\sigma^2}+Bx \qquad \text{ y } v'(x)=-1=Ax^{-2r/\sigma^2-1}\left(\frac{-2r}{\sigma^2}\right)+B, 
          \]
          de tal forma que resolviendo para $x$, se tiene que $x=\frac{2r}{2r+\sigma^2}K$, justo como 
          se buscaba.
         \end{proof}
    \end{enumerate}
    \item[\textbf{4.}] \textbf{(Put Americano perpetuo que paga dividendos).} Considere un put Americano perpetuo 
    con base en un movimiento Browniano geométrico pagando dividendos a tasa constante $a>0$. El diferencial 
    de este activo está dado por 
    \[
    dS(t)=(r-a)S(t)dt+\sigma(t)d\widetilde{W}(t), 
    \]
    donde $\widetilde{W}(t)$ es un movimiento Browniano bajo la medida de riesgo neutral $\widetilde{\P}$.
    \begin{enumerate}
        \item Suponga que adoptamos la estrategia de ejercer el put oen el primer instante 
        que el precio alcanza o está por debajo de $L$. ¿Cuál es el payoff descontado 
        esperado bajo la medida de riesgo neutral de esta estrategia (tal como lo hicimos en clase 
        sin dividendos)? Escriba esto como una función $v_L(x)$ del precio inicial del activo $x$. 
        (Sugerencia: defina la constante positiva 
        \[
            \gamma=\frac{1}{\sigma^2}\left(r-a-\frac{1}{2}\sigma^2\right)+\frac{1}{\sigma}\sqrt{\frac{1}{\sigma^2}\left(r-a-\frac{1}{2}\sigma^2\right)^2+2r}
        \]
        y escriba $v_L(x)$ usando $\gamma$).
        \begin{proof} 
          Resolviendo la ecuación diferencial $dS(t)=(r-a)S(t)dt+\sigma S(t)d\widetilde{W}(t)$, tenemos que 
          
          \[
          S(t)=S(0)\exp \left\{\sigma \widetilde{W}(t)+(r-a-\frac{1}{2}\sigma^2)t\right\}  
          \]
          Denotando por $\alpha:=r-a-\frac{1}{2}\sigma^2$, notamos que 
          la función $v_L(x)$ debe estar dada por $K-x$ si $x\in [0,L]$ (justamente ejercemos la opción 
          si es que desde un inicio la opción al tiempo 0 es $S(0)=x\leq L$), y buscamos ahora 
          la expresión para la opción cuando $S(0)=x>L$. Nótese que $S(t)=L$ si y solo si
          \[
            x\exp \left\{\sigma\widetilde{W}(t)+\alpha t\right\}=L,
          \]
          lo cual ocurre si y solo si 
          \[
          -\sigma\widetilde{W}(t)-\frac{\alpha}{\sigma} t=\frac{1}{\sigma}\ln(\frac{x}{L}), 
          \]
          y aplicamos el teorema visto en clase para el movimiento browniano $-\widetilde{W}(t)$ y 
          el término de deriva $-\frac{\alpha}{\sigma} t$. En este caso usamos 
          que $\lambda=r$, $\mu=-\frac{\alpha}{\sigma}$ y $m=\frac{1}{\sigma}\ln \left(\frac{x}{L}\right)$. Esto nos da como resultado
            que 
            \[
            \widetilde{\E}\left[e^{-r\tau_m}\right]=e^{-m(-\mu+\sqrt{\mu^2+2\lambda})}    
            \]
            lo cual sustituyéndolo por las cantidades mencionadas antes, nos origina que 
            \[
            \widetilde{\E}\left[e^{-r\tau_L}\right]=\exp \left\{-\frac{1}{\sigma}\ln \left(\frac{x}{L}\right)(\frac{\alpha}{\sigma}+\sqrt{\frac{\alpha^2}{\sigma^2}+2r})\right\}=\exp \left\{-\ln \left(\frac{x}{L}\right)(\frac{\alpha}{\sigma{2}}+\frac{1}{\sigma}\sqrt{\frac{\alpha^2}{\sigma^2}+2r})\right\}=\left(\frac{x}{L}\right)^{-\gamma},
            \]
            y esta última es la expresión del precio inicial del activo $S(0)=x$.
         \end{proof}
        \item Determine $L_*$ el valor de $L$ que maximiza el payoff descontado esperado 
        bajo la medida de riesgo neutral $v_L(x)$ calculado en el ítem anterior.
        \begin{proof} 
          Para esto, tomamos la expresión $v(S(0))=v(x)=(K-L)(\frac{x}{L})^{-\gamma}$ y derivamos en términos de $L$ para 
          obtener el máximo. Fijamos el valor de $x$ (pues el máximo en $L$ debe funcionar para todos los $x$), y notamos que 
          \[
          g(L)=(K-L)L^{\gamma}  
          \]
          debe ser maximizada. Pero esta función encuentra su máximo en 
          \[
          g'(L)=\gamma(K-L)L^{\gamma-1}-L^{gamma}=0   \iff \gamma(K-L)=L \iff L=\frac{\gamma K}{\gamma+1},
          \]
          y por lo tanto $L=\frac{\gamma}{\gamma+1}K$ es la barrera óptima a elegir para ejercer el Put.



         \end{proof}
        \item Muestre que para cualquier precio inicial $S(0)=x$ del activo, el proceso $e^{-rt}v_{L_*}(S(t))$ es una 
        supermartingala bajo $\widetilde{\P}$. Muestre que si $S(0)=x>L_*$ y $e^{-rt}v_{L_*}(S(t))$ es 
        parado al primer instante en el que el precio del activo llega a $L_*$, entonces 
        la supermartingala parada es una martingala. (Sugerencia: muestre que 
        \[
        r+(r-a)\gamma-\frac{1}{2}\sigma^2\gamma(\gamma+1)=0).  
        \]
        \item Muestre que, para cualquier precio inicial del activo $S(0)=x$, 
        \[
        v_{L_*}(x)=\max_{\tau \in \mathcal{T}}\widetilde{\E}\left[e^{-r\tau}(K-S(\tau))\right],    
        \]
        donde $\mathcal{T}$ es el conjunto de todos los tiempos de paro.
    \end{enumerate}
    \end{itemize}
\end{document}
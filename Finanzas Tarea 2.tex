\documentclass[letterpaper]{article} 
\usepackage[left = 0.5in, right = 0.5in, top = 0.9in, bottom = 0.9in]{geometry}
\usepackage{enumitem}
\usepackage{multicol}
\usepackage[spanish]{babel}
\usepackage[utf8]{inputenc}

\usepackage{amsmath,amssymb,amsthm}
\usepackage{tikz-cd}
\usepackage{mathrsfs}
\usepackage[bbgreekl]{mathbbol}
\usepackage{dsfont}
\usepackage{graphicx}
\graphicspath{{img/}}


\newcommand{\op}{\operatorname}
\newcommand{\Op}{^{\op{op}}}
\newcommand{\scc}{\mathscr C}
\newcommand{\scd}{\mathscr D}
\newcommand{\sce}{\mathscr E}
\newcommand{\sci}{\mathscr I}
\newcommand{\scj}{\mathscr J}
\newcommand{\scx}{\mathscr X}
\newcommand{\var}{\mathrm{Var}}
\newcommand{\Id}{\operatorname{Id}}
\newcommand{\N}{\mathbb N}
\newcommand{\Z}{\mathbb Z}
\newcommand{\Q}{\mathbb{Q}}
\newcommand{\I}{\mathbb{I}}
\newcommand{\R}{\mathbb{R}}
\newcommand{\C}{\mathbb{C}}
\newcommand{\F}{\mathcal{F}}
\newcommand{\G}{\mathcal{G}}
\newcommand{\B}{\mathcal{B}}
\newcommand{\abs}[1]{\left\lvert #1 \right\rvert}
\newcommand{\inv}{^{-1}}
\renewcommand{\to}{\rightarrow}
\newcommand{\ent}{\Longrightarrow}
\newcommand{\E}{\mathbb{E}}
\renewcommand{\P}{\mathbb{P}}
\newcommand{\1}{\mathds{1}}
\renewcommand{\qedsymbol}{$\blacksquare$}

\theoremstyle{definition}
\newtheorem{dfn}{Definición}
\theoremstyle{definition}
\newtheorem{teo}{Teorema}
\theoremstyle{definition}
\newtheorem{cor}{Corolario}
\theoremstyle{definition}
\newtheorem{prop}{Proposición}
\theoremstyle{definition}
\newtheorem{obs}{Observación}


\title{\textbf{Modelos Estocásticos en Finanzas\\
Tarea 2}}
\author{Iván Irving Rosas Domínguez}
\date{\today}

\DeclareSymbolFontAlphabet{\mathbbm}{bbold}
\DeclareSymbolFontAlphabet{\mathbb}{AMSb}
\DeclareMathSymbol\bbDelta  \mathord{bbold}{"01}

\begin{document}
\maketitle

%\begin{abstract}
%\end{abstract}
\textbf{Problema 1.-} Considere el problema de inversión óptima visto en clase, dado por: dado $X_0$, encuentre el proceso adaptado $\Delta_0,...,\Delta_{N-1}$
que maximiza
\[
\E\left[U(X_N)\right],    
\]
sujeto a la condición 
\[
X_{n+1}=\Delta_nS_{n+1}+(1+r)(X_n-\Delta_nS_n), \quad n=0,1,...,N-1.    
\]
\begin{enumerate}
    \item[\textbf{1.}] Considere el Problema 1 para un modelo $N-$binomial con función de utilidad $U(x)=\ln(x)$. Muestre que el proceso de la riqueza óptima
    correspondiente al proceso de portafolio óptimo está dado por $X_n=\frac{X_0}{\zeta_n}$, $n=0,...,N-1$, donde $\zeta_n$ es el proceso de 
    densidad de precio visto en clase dado por
    \[
    \zeta_n=\frac{Z_n}{(1+r)^{N}}.    
    \]
    \begin{proof} 
       En clase se vio que el problema 1 podía bien ser reducido al siguiente problema:

       \textbf{Problema 1.1.-} Dado $X_0$, hallar un vector $(x_1,...,x_M)$ que maximiza 
       \[
       \sum_{m=1}^{M}p_mU(x_m), 
       \] 
       sujeto a la restricción 
       \[
       \sum_{m=1}^M p_mx_m\zeta_m=X_0,
       \]
       donde $p_m=\P(\omega^m)$, $x_m=X_N(\omega^m)$, $\zeta_m=\zeta(\omega^{m})$ y $\omega^m$ es una de las $2^{M}$ posibles sucesiones
       de volados en un modelo $N$-binomial a $N$ periodos, con $M=2^{M}$.\\
       
       El problema anterior es equivalente al problema inicial y puede ser resuelto utilizando 
       multiplicadores de Lagrange. Para ello, se dedujo que el lagrangiano $L$ era 
       \[
        L=\sum_{m=1}^{M}p_mU(x_m)-\lambda \left(\sum_{m=1}^Mx_m\zeta_mp_m-X_0\right),
       \]   
       de tal forma que $U'(X_N)=\lambda\zeta_N=\frac{\lambda Z_N}{(1+r)^N},$ donde $Z$ es la
       derivada de Radon-Nikodým de $\tilde{\P}$ la medida de riesgo neutral con respecto a $\P$ la medida
       de riesgo real, y $Z_n=\E\left[Z|F_n\right]$ es el n-ésimo elemento del proceso derivada de Radón-Nikodým.
       
       Recordando que $U$ es una función que suponemos cóncava, su derivada es monótona creciente y
       por lo tanto tiene una inversa. Luego, 
       \[
       X_N=I \left(\frac{\lambda Z_n}{(1+r)^N}\right),
       \]
       donde $I$ es la función inversa de $U'$, y se encuentra la solución primero hallando
       $\lambda$ a partir de la ecuación 
       \[
       X_0=\E\left[\frac{Z_n}{(1+r)^N}I \left(\frac{\lambda Z_n}{(1+r)^N}\right)\right] 
       \]
       y posteriormente hallando $X_N$ con la ecuación 
       \[
        X_N=I \left(\frac{\lambda Z_N}{(1+r)^N}\right).
       \]
       Dado que en nuestro caso, tenemos que $U(x)=\ln(x)$, la cual es claramente
       una función cóncava, se deduce que $U'(x)=\frac{1}{x}$ y con ello $I(y)=\frac{1}{y}$.
       Luego, dado que nosotros ya estamos suponiendo que hay un proceso de portafolio que 
       alcanza riqueza óptima $X_N$, entonces se deben de cumplir las condiciones del problema $1.1$ 
       anteriormente escrito. Hallamos pues $\lambda$ de la primera de las dos ecuaciones anteriores:
        \begin{align*}
            X_0=\E\left[\frac{Z_n}{(1+r)^N}I \left(\frac{\lambda Z_n}{(1+r)^N}\right)\right]&=\E\left[\frac{Z_n}{(1+r)^N} \left(\frac{(1+r)^N}{\lambda Z_n}\right)\right]\\
            &=\E\left[\frac{1}{\lambda}\right]=\frac{1}{\lambda},
        \end{align*}
        y una vez que tenemos lo anterior, calculamos $X_N$:
        \[
        X_N= I \left(\frac{\lambda Z_n}{(1+r)^N}\right)=\frac{(1+r)^{N}}{\lambda Z_n}=\frac{X_0}{(1+r)^N},
        \]
        de tal forma que para $N$, el valor óptimo de la riqueza está dado por 
        \[
        X_N=\frac{X_0}{(1+r)^{N}},    
        \]
        en este caso en el que $U(x)=\ln(x)$. Luego, para calcular el resto del proceso de riqueza, hacemos
        uso de que el proceso 
        \[
        \left(\frac{X_n}{(1+r)^n}\right)_{n\geq0},    
        \]
        es una $\tilde{P}$-martingala con respecto a la filtración generada por los volados. Tenemos 
        así que 
        \begin{align*}
            \frac{X_n}{(1+r)^n}&=\tilde{\E}\left[\frac{X_N}{(1+r)^N}\big|\F_n\right]\\
            &=\tilde{\E}\left[\frac{X_0}{\zeta_N(1+r)^N}\big|\F_n\right]\\
            &=\tilde{\E}\left[\frac{X_0\cdot (1+r)^N}{Z_N(1+r)^N}\big|\F_n\right]\\
            &=X_0\tilde{\E}\left[\frac{1}{Z_N}\big|\F_n\right],\\
        \end{align*}
        por lo que resta investigar el comportamiento de la última esperanza condicional.
        Por definición, sabemos que $Z_N=Z$ es la derivada de Radón-Nikodým de $\tilde{\P}$ con respecto
        a $\P$, la cual, en nuestro contexto, está definida para $\omega^{m}$ alguna de todas las 
        posibles sucesiones de volados, como
        \[
            Z(\omega^m)=\frac{\tilde{\P}(\omega^m)}{\P(\omega^m),}
        \]
        La variable está bien definida pues las probabilidades son siempre positivas, y además, se vio en clase que
        $\tilde{\E}\left[Z\right]=1$ y que $\tilde{\E}\left[Y\big|\F_n\right]=\frac{1}{Z_n}\E\left[Z_mY\big|\F_n\right]$, para 
        $Y\in \F_n$. Obsérvese que podemos definir a la variable $Z':=\frac{1}{Z}$, de tal forma que
        \[
        Z'(\omega^{m})=\left(\frac{\tilde{\P}(\omega^m)}{\P(\omega^m)}\right)^{-1}=\frac{\P(\omega^m)}{\tilde{\P}(\omega^m)},
        \]
        la cual es la derivada de Radón-Nikodým de $\P$ con respecto a $\tilde{\P}$. Y esta variable va a cumplir exactamente
        las mismas condiciones que $Z$, así como su proceso derivada. En particular, tendremos que 
        \[
            \tilde{\E}\left[\frac{1}{Z_N}\big|\F_n\right]=\tilde{\E}\left[\frac{1}{Z}\big|\F_n\right]=\tilde{\E}\left[Z'\big|\F_n\right]=Z'_n=\frac{1}{Z_n},
        \]
        por lo que volviendo a nuestra ecuación de interés, se sigue que 
        \[
            \frac{X_n}{(1+r)^n}=X_0\tilde{\E}\left[\frac{1}{Z_N}\big|\F_n\right]=\frac{X_0}{Z_n},
        \]
        por lo que despejando, se tiene que para cualquier $n\in \{0,1,...,N-1\}$,
        \[
        X_n=\frac{X_0\cdot(1+r)^n}{Z_n}=\frac{X_0}{\zeta_n},    
        \]
        tal y como buscábamos.
    \end{proof}
    \item[\textbf{2.}] Considere el Problema 1 para un modelo $N$-binomial con función de utilidad
    $U(x)=\frac{1}{p}x^{p}$ con $p<1$ y $p\neq0$. Muestre que la riqueza óptima al tiempo 
    $N$ está dada por
    \[
    X_N=\frac{X_0(1+r)^{N}Z^{\frac{1}{p-1}}}{\E\left[Z^{\frac{p}{p-1}}\right]},    
    \]
    donde $Z$ es la derivada de Radon-Nikodým de $\tilde{\P}$ con respecto a $\P$.
    \begin{proof} 
          
      Nuevamente utilizamos el problema $1.1$ para resolver el problema original.
      Notamos que la función de aversión al riesgo $U(x)=\frac{1}{p}x^{p}$ es una función cóncava, ya que
      el exponente $p$ es menor a 1 y en particular no es 0 en ningún momento. Incluso cuando $p<0$, la función
      sigue siendo cóncava. Notamos también que 
      \[
      U'(x)=\frac{p}{p}x^{p-1}=x^{p-1}, \qquad p<1, p\neq 0,    
      \]
      y esta función es invertible en $(0,\infty)$, y tiene por inversa a 
      \[
      I(y)=y^{\frac{1}{p-1}}, \qquad p<1, p\neq 0.
      \]
      Por lo tanto, utilizando las ecuaciones
      \[
        X_N=I \left(\frac{\lambda Z_N}{(1+r)^N}\right),
        \]
        y
        \[
        X_0=\E\left[\frac{Z_N}{(1+r)^N}I \left(\frac{\lambda Z_N}{(1+r)^N}\right)\right],
        \]
        vistas en clase y ajustadas a este problema, encontramos el multiplicador $\lambda$:
        \begin{align*}
            X_0=\E\left[\frac{Z_N}{(1+r)^N}I \left(\frac{\lambda Z_N}{(1+r)^N}\right)\right]&=\E\left[\frac{Z_N}{(1+r)^N}\left(\frac{\lambda Z_n}{(1+r)^N}\right)^{\frac{1}{p-1}}\right]\\
            &=\E\left[\frac{\lambda^{\frac{1}{p-1}}Z_N^{1+\frac{1}{p-1}}}{(1+r)^{N+\frac{N}{p-1}}}\right]\\
            &=\frac{\lambda^{\frac{1}{p-1}}}{(1+r)^{\frac{Np}{p-1}}}\E\left[Z_N^{\frac{1}{p-1}}\right]\\
            &=\frac{\lambda^{\frac{1}{p-1}}}{(1+r)^{\frac{Np}{p-1}}}\E\left[Z^{\frac{1}{p-1}}\right]\\
        \end{align*}
        donde hemos usado que $Z_N=Z$, ya que justamente el proceso derivada de Radón-Nikodým es creado
        a partir de la martingala de Doob con base en la variable $Z$, a saber, $Z_n=\E\left[Z|\F_n\right]$, por lo 
        que en particular $Z=\E\left[Z|\F_N\right]=Z_N$, ya que $Z$ es $\F_N$-medible, pues es una variable
        que depende de todos los valores de los volados desde $0$ hasta $N$.\\

        Luego, despejando $\lambda$, tenemos que 
        \[
        \lambda^{\frac{1}{p-1}}=\frac{X_0(1+r)^{\frac{Np}{p-1}}}{\E\left[Z^{\frac{1}{p-1}}\right]}.   
        \]
        Ya con el valor de $\lambda$, utilizamos la ecuación que nos entrega el valor de la riqueza en el instante $N$, a saber,
        \begin{align*}
            X_N=I \left(\frac{\lambda Z}{(1+r)^N}\right)&=\left(\frac{\lambda Z}{(1+r)^N}\right)^{\frac{1}{p-1}}\\
            &=\lambda^{\frac{1}{p-1}}\frac{Z^{\frac{1}{p-1}}}{(1+r)^{\frac{N}{p-1}}}\\
            &=\frac{X_0(1+r)^{\frac{Np}{p-1}}}{\E\left[Z^{\frac{1}{p-1}}\right]}\frac{Z^{\frac{1}{p-1}}}{(1+r)^{\frac{N}{p-1}}}\\
            &=\frac{X_0(1+r)^{\frac{Np}{p-1}-\frac{N}{p-1}}Z^{\frac{1}{p-1}}}{\E\left[Z^{\frac{1}{p-1}}\right]}\\
            &=\frac{X_0(1+r)^{N}Z^{\frac{1}{p-1}}}{\E\left[Z^{\frac{p}{p-1}}\right]},    
        \end{align*}
        tal y como se quería.

        \end{proof}
    \item[\textbf{3.}] Un inversionista provee una pequeña cantidad de dinero $X_0$ para que se pruebe la efectividad de 
    una estrategia de inversión sobre los siguientes $N$-periodos. Puedes invertir en el modelo $N$-binomial
    sujeto a la condición de que el portafolio nunca puede ser negativo. Si al tiempo $N$ el valor del portafolio
    es a lo menos $\gamma$, una constante positiva especificada por el inversionista, 
    entonces él te otorgará una gran cantidad de dinero para que se la manejes. Entonces el problema 
    es el siguiente: 

    Maximizar 
    \[
    \P\left(X_N\geq\gamma\right),
    \]
    donde $X_N$ es el portafolio comenzando on fortuna $X_0$ bajo la condición de que 
    \[
    X_n\geq0, \qquad n\in \{0,1,...,N\}.    
    \]
    Podemos reformular el problema como sigue: maximizar 
    \[
    \P\left(X_N\geq \gamma\right)    
    \]
    sujeto a la restricción
    \[
        \tilde{\E}\left[\frac{X_N}{(1+r)^N}\right]=X_0, \qquad X_n\geq0, \qquad n\in \{0,1,...,N\}.
    \]
    \begin{enumerate}
        \item Muestra que si $X_N\geq0$, entonces $X_n\geq0$ para todo $n\geq0$.
        \begin{proof} 
          Recordamos nuevamente que el proceso $\left(\frac{X_n}{(1+r)^{n}}\right)_{n\geq0}$ es martingala
          bajo la medida de riesgo neutral. Luego, dado que por hipótesis estamos suponiendo que $X_N$ es 
          una variable aleatoria que representa la riqueza al tiempo $N$ acorde a cierto proceso de portafolio,
          y esta es positiva, entonces por
          monotonía de la esperanza condicional y el hecho de que $1+r>0$, se tiene que
          \[
          0\leq\tilde{\E}\left[\frac{X_N}{(1+r)^{N}}\big|\F_n\right]=\frac{X_n}{(1+r)^{n}}, \qquad n\in \{0,1,...,N-1\},
          \] 
          por lo que para cualquier $n\in \{0,1,...,N-1\}$ se sigue que 
          \[
          0\leq X_n  
          \]
         \end{proof}
        \item Considera la función 
        \begin{equation*}
            U(x) = \begin{cases}
                    0, & \text{si } 0\leq x\leq \gamma,\\
                    1, & \text{si } x\geq \gamma.
                    \end{cases}
        \end{equation*}
    Muestra que para cada $y>0$ fija, se tiene que $$U(x)-yx\leq U(I(y))-yI(y)$$    
    para toda $x\geq0$, donde 
    \begin{equation*}
        I(y)=\begin{cases}
            \gamma, & \text{si } 0<y\leq \frac{1}{\gamma},\\
            0, & \text{si } \frac{1}{\gamma}.
        \end{cases}
    \end{equation*}
    \begin{proof} 
      Observamos que las funciones $U$ e $I$ pueden ser escritas como: $U:[0,\infty)\to \R$, $U(x)=\1_{[0,\frac{1}{\gamma}]}(x)$ y  $I:(0,\infty)\to \R$, $I(y)=\gamma\1_{[\gamma,\infty)}(y)$.
      Sean $y>0$ y $x\geq0$. Procedemos a probar la desigualdad anterior por casos:
      \begin{itemize}
        \item[$0< y \leq \frac{1}{\gamma}$:] en este caso, $I(y)=\gamma$, por lo que 
        \begin{align*}
            U(x)-yx\leq U(I(y))-yI(y)&\Longleftrightarrow\1_{[\gamma,\infty)}(x)-yx\leq U(\gamma)-y\gamma\\
            &\Longleftrightarrow \1_{[\gamma,\infty)}(x)-yx\leq \1_{[\gamma,\infty)}(\gamma)-y\gamma\\
            &\Longleftrightarrow \1_{[\gamma,\infty)}(x)+y\gamma\leq 1+yx.
        \end{align*}
        Luego, si $x<\gamma$, entonces la indicadora de la izquierda en la expresión anterior es 0 y 
        \[
            U(x)-yx\leq U(I(y))-yI(y) \Longleftrightarrow y\gamma\leq 1+yx,
        \]
        pero dado que $0<y<\frac{1}{\gamma}$, se tiene que $y\gamma\leq 1\leq 1+yx$, pues $y>0$ y $x\geq0$, por lo que
        la desigualdad se da. Por otro lado, si $x\leq\gamma$, entonces 
        \[
            U(x)-yx\leq U(I(y))-yI(y) \Longleftrightarrow 1+ y\gamma\leq 1+yx \Longleftrightarrow \gamma\leq x,
        \]
        lo cual es justamente lo que estamos suponiendo. Por lo tanto, la desigualdad se da nuevamente. 
        \item[$y\geq\frac{1}{\gamma}$:] en este caso, $I(y)=0$ y por lo tanto, 
        \begin{align*}
            U(x)-yx\leq U(I(y))-yI(y)&\Longleftrightarrow U(x)-yx\leq U(0)-y\cdot0\\
            &\Longleftrightarrow \1_{[\gamma,\infty)}(x)-yx\leq \1_{[\gamma,\infty)}(0)\\
            &\Longleftrightarrow \1_{[\gamma,\infty)}(x)\leq yx,
        \end{align*}
        por lo que si $x<\gamma$, entonces 
        \[
            U(x)-yx\leq U(I(y))-yI(y)\Longleftrightarrow \1_{[\gamma,\infty)}(x)\leq yx \Longleftrightarrow 0\leq xy,
        \] 
        pero esto último ocurre pues $x\geq0$ y $y>0$. Por otro lado, si $x\geq\gamma$, entonces
        \[
            U(x)-yx\leq U(I(y))-yI(y)\Longleftrightarrow \1_{[\gamma,\infty)}(x)\leq yx \Longleftrightarrow1\leq xy,
        \]
        pero dado que $y\geq \frac{1}{\gamma}$, entonces $xy\geq 1$, por lo que la desigualdad ocurre en ambos casos.
      \end{itemize}
     \end{proof}
    \item Suponga que existe una solución $\lambda$ a la ecuación
           \begin{equation}\label{ec1}
             \E\left[\frac{Z_N}{(1+r)^N}I \left(\frac{\lambda Z_N}{(1+r)^N}\right)\right]=X_0.
           \end{equation}
    Muestre que la $X_N$ óptima está dada por (como se hizo en clase)
    \[
        X_N^{*}=I \left(\frac{\lambda Z}{(1+r)^{N}}\right).
        \]
        \begin{proof} 
          Utilizamos el inciso anterior. Notamos que haciendo $x=X_N$ una solución cualquiera al problema anterior, no 
          necesariamente la óptima, pero la cual es mayor
          o igual a 0 pues buscamos que se cumplan las condiciones del problema, y haciendo $y=\frac{\lambda Z}{(1+r)^{N}}$, el cual 
          también es un valor mayor a 0, se deduce que
          \[
            U(x)-yx\leq U(I(y))-yI(y)\Longleftrightarrow U(X_N)-X_N\frac{\lambda Z}{(1+r)^N} \leq U\left(I\left(\frac{\lambda Z}{(1+r)^N}\right)\right)-\frac{\lambda Z}{(1+r)^N}I \left(\frac{\lambda Z}{(1+r)^N}\right),
          \]
          por lo que tomando esperanzas, y recordando la definición de $X_N^*$, de $X_0$, que $\lambda$ es solución a la ecuación \eqref{ec1} y que $Z$ es la derivada 
          de Radon-Nikodým de $\tilde{\P}$ con respecto a $\P$, 
          
          \begin{align*}
            \E\left[U(X_N)\right] -&\E\left[X_N\frac{\lambda Z}{(1+r)^N}\right] \leq \E\left[U\left(I\left(\frac{\lambda Z}{(1+r)^N}\right)\right)\right] - \E\left[\frac{\lambda Z}{(1+r)^N}I \left(\frac{\lambda Z}{(1+r)^N}\right)\right]\\
            &\Longleftrightarrow \E\left[U(X_N)\right] -\lambda\E\left[X_N\frac{Z}{(1+r)^N}\right] \leq \E\left[U\left(I\left(\frac{\lambda Z}{(1+r)^N}\right)\right)\right] - \lambda\E\left[\frac{Z}{(1+r)^N}I \left(\frac{\lambda Z}{(1+r)^N}\right)\right]\\
            &\Longleftrightarrow \E\left[U(X_N)\right] -\lambda\tilde{\E}\left[\frac{X_N}{(1+r)^N}\right] \leq \E\left[U\left(X_N^*\right)\right] - \lambda X_0\\
            &\Longleftrightarrow \E\left[U(X_N)\right] -\lambda X_0 \leq \E\left[U\left(X_N^*\right)\right] - \lambda X_0\\
            &\Longleftrightarrow \E\left[U(X_N)\right] \leq \E\left[U\left(X_N^*\right)\right]\\
            &\Longleftrightarrow \E\left[\1_{[\gamma,\infty)}(X_N)\right]\leq \E\left[\1_{[\gamma,\infty)}(X_N^{*})\right]\\
            &\Longleftrightarrow \P\left(X_N\geq \gamma\right) \leq \P\left(X_N^{*}\geq \gamma\right),
          \end{align*}
          tal y como queríamos. Por lo tanto, la variable $X_N^*$ dada por la expresión anterior en efecto maximiza 
          la probabilidad buscada.
        \end{proof}
    \item Si en listamos las $M=2^{N}$ posibles secuencias de volados y las etiquetamos por $\omega^1,...,\omega^M$ y 
    definimos $\zeta_m=\zeta(\omega^{m})$ y $p_m=\P(\omega^m)$. Aquí enlistamos las $\omega_m's$ de 
    manera que las $\zeta_m's$ sean ascendentes, es decir,
    \[
     \zeta_1\leq\zeta_2\leq\dots\leq \zeta_M.    
    \]
    Muestre que la suposición de que existe $\lambda$ solución a \eqref{ec1} es equivalente a suponer
    que para algún entero positivo $K$ tenemos que $\zeta_K<\zeta_{K+1}$ y 
    \[
    \sum_{m=1}^{K}\zeta_mp_m=\frac{X_0}{\gamma}.    
    \]
    \begin{proof} 
      Por definición de esperanza y de $\zeta_m$, si suponemos que $\lambda$ es solución a \eqref{ec1}, entonces
      \[
     X_0=\E\left[\frac{Z_N}{(1+r)^N}I \left(\frac{\lambda Z_N}{(1+r)^N}\right)\right]=\E\left[\zeta_N I(\lambda\zeta_N)\right]=\E\left[\zeta I(\lambda\zeta)\right]=\sum_{m=1}^{M}\zeta_mI(\lambda\zeta_m)p_m,
      \] 
      ahora bien, como $X_0$ lo suponemos estrictamente positivo, entonces 
      \[
        \sum_{m=1}^{M}\zeta_mI(\lambda\zeta_m)p_m>0,
      \]
      por lo que al ser todos los sumandos positivos, al menos existe un elemento en el conjunto $A=\{m\in \{1,...,M\} : \zeta_mI(\zeta_m\lambda)p_m>0\}$.
      Dado que $A$ es un subconjunto no vacío de $\N$ que está acotado, entonces tiene máximo. Sea $K=\max A$. Nótese que 
      \[
      \zeta_KI(\zeta_K\lambda)p_K>0 \ent I(\zeta_K\lambda)=\gamma\neq 0,
      \]
      de donde se deduce que $0<\zeta_K\lambda\leq \frac{1}{\gamma}$ por definición de $I$. Pero lo anterior se traduce en que
      \[    
        \zeta_K\leq \frac{1}{\lambda \gamma},
        \]  
        por lo que, como los $\zeta_m$ están ordenados de menor a mayor, se tiene que
        para cualquier $m\in A$ con $m\leq K$, 
        \[
        \zeta_m\leq \frac{1}{\lambda \gamma},    
        \]
        y con ello, 
        \[
        \zeta_mI(\zeta_m\lambda)p_m=\zeta_m\gamma p_m, \qquad \forall \ 1\leq m \leq K,    
        \]
        así como para cualquier $m\in A$ con $m>K$, por el mismo orden de los $\zeta_m$, 
        $$\zeta_mI(\zeta_m\lambda)p_m=0 \ent I(\zeta_m\lambda)=0 \ent \zeta_m\lambda>\frac{1}{\gamma} \ent \zeta_m>\frac{1}{\lambda \gamma},$$
        así que en particular para $m=K+1$,
        \[
        \zeta_K\leq\frac{1}{\lambda\gamma}<\zeta_{K+1},    
        \]
        es decir, $\zeta_K<\zeta_{K+1}$.

        Por lo tanto, 
        \[
        X_0=\E\left[\zeta \left(\lambda\zeta\right)\right]=\sum_{m=1}^{M}\zeta_mI(\lambda\zeta_m)p_m=\sum_{m=1}^{K}\zeta_mI(\zeta_m\lambda)p_m=\sum_{m=1}^{K}\zeta_m\gamma p_m,   
        \]
        de donde se sigue que 
        \[
        \sum_{m=1}^{K}\zeta_mp_m=\frac{X_0}{\gamma}.    
        \]
        De manera contraria, si se tiene que existe $K\in \{1,...,M\}$ tal que 
        \[
        \sum_{m=1}^{K}\zeta_mp_m=\frac{X_0}{\gamma},    
        \]
        con $\zeta_K<\zeta_{K+1}$, entonces podemos elegir $\lambda>0$ una constante positiva
        tal que 
        \[
        \lambda\zeta_K<\frac{1}{\gamma},
        \]
        de tal forma que, como $\lambda\zeta_K\leq \frac{1}{\gamma}<\lambda\zeta_{K+1}$, con lo 
        que $I \left(\lambda\zeta_K\right)=\gamma$ y por el orden en los $\zeta_m$, se tiene que 
        \begin{equation}
        I(\lambda\zeta_m)=\begin{cases}\label{ec2}
            \gamma, & \text{si } \lambda\zeta_m\leq \frac{1}{\gamma} \Longleftrightarrow m\leq K,\\
            0, & \text{si } \lambda\zeta_m > \frac{1}{\gamma} \Longleftrightarrow m>K.
        \end{cases}    
        \end{equation}
        Por lo que 
        \[
        X_0=\sum_{m=1}^{K}\zeta_m\gamma p_m=\sum_{m=1}^K\zeta_mI(\lambda\zeta_m)p_m=\sum_{m=1}^M\zeta_mI(\lambda\zeta_m)p_m=\E\left[\zeta I(\lambda\zeta)\right]=\E\left[\frac{Z_N}{(1+r)^N}I \left(\frac{\lambda Z_N}{(1+r)^N}\right)\right],
        \]
        y concluimos que $\lambda$ es solución de 
        \[
            \E\left[\frac{Z_N}{(1+r)^N}I \left(\frac{\lambda Z_N}{(1+r)^N}\right)\right]=X_0.    
        \]

     \end{proof}
    \item Muestre que $X_N^*$ está dado por 
    \[
    X_N^{*}(\omega^{m})=\begin{cases}
        \gamma, & \text{si }m\leq K,\\
        0, & \text{si } m\geq K+1.
    \end{cases}    
    \]
    \begin{proof} 
       Del inciso $c)$ y $d)$, y particularmente de \eqref{ec2} tenemos que para $m\in \{1,...,M\}$, 
       \[
       X_N^{*}(\omega^{m})=I(\lambda\zeta)(\omega^{m})=I(\lambda\zeta_m)=\begin{cases}
        \gamma, & \text{si } m\leq K,\\
        0, & \text{si } m>K.
       \end{cases} 
       \]
     \end{proof}
    \end{enumerate}
\end{enumerate}
\end{document}